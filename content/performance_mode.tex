%Depending on visible amount, more or less complex procedures are used.

Performance Modes bilden ein Zusammenspiel aus der Anzahl der sichtbaren Knoten, also derjenigen, die in-bounds sind, und den
genutzten Labeling Verfahren. Allgemein gesagt: Je mehr Knoten sichtbar sind, desto weniger Verfahren werden genutzt -
und die genutzten Verfahren sind die einfacheren Verfahren.

Dies wird durch Grenzwerte realisiert (siehe \hyperref[subsec:consts]{Magic Constants}).
Die Grenzwerte (siehe \autoref{fig:perf_modes}) wurden vor allem durch manuelles Testen ermittelt.
Ziel war es das eine Iteration des Render-Loops möglichst unter $50$ ms bleibt.

\begin{figure}[H]
    \centering
    $$
    \begin{array}{c|c|l}
        \text{Minimum} & \text{Maximum} & \text{Verfahren}\\ \hline
        1 & 99 & \text{alle Verfahren}\\ \hline
        100 & 499 & \text{\hyperref[subsubsec:4pos]{4-Position-Model}, \hyperref[subsubsec:8pos]{8-Position-Model}, \hyperref[subsubsec:slider]{Slider-Model} }\\ \hline
        500 & 999 & \text{\hyperref[subsubsec:4pos]{4-Position-Model}, \hyperref[subsubsec:8pos]{8-Position-Model} }\\ \hline
        500 & \infty & \text{\hyperref[subsubsec:4pos]{4-Position-Model}}
    \end{array}
    $$
    \caption{Intervalle wie standardmäßig im Projekt festgelegt (siehe \hyperref[subsec:consts]{Magic Constants}) mit dazugehörigen Labeling Verfahren}
    \label{fig:perf_modes}
\end{figure}

Hier ist der Zweck, dass man bei sehr vielen sichtbaren Knoten durch das Labeling eine schnelle, grobe Orientierung bekommen kann.
Danach kann man via Zoom-In den gewünschten Bereich genauer explorieren, wofür dann auch ein aufwendiges und somit detailliertes Labeling wünschenswert wäre.
Da dann auch weniger Knoten zu sehen sind, was zu weniger Vergleichen führt, ist hier genug Zeit für solche komplizierten Berechnung.
