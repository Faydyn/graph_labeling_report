To tackle the NP-hard problem of graph labeling, a greedy approach was used to generate label positions.
Conflict Detection is done fastly with simple bounding boxes, so that valid labels can be stored quick and efficient in proper data structures like maps and quadtrees.
Animations are used to ease changes of label positions or visibility on the eye.
All these techniques combined make for a fluid and visually appealing experience, that does justice to dynamic labeling.\\
\\
Um das NP-Harte Problem des Graph Labeling in dynamischen Graphen anzugehen, wird ein gieriger Ansatz genutzt, um Label Position zu generieren.
Mit simplen Bounding Boxes wird schnelle Konfliktermittlung betrieben, sodass valide Positionen durch geeignete Datenstrukturen wie Maps und Quadtrees effizient und zügig gespeichert werden.
Animationen runden Änderungen der Labelpositionen oder -sichtbarkeit visuell ab.
Diese Maßnahmen führen im Gesamtbild zu einem flüssigen, visuell ansprechenden Ergebnis, das der Natur von dynamischem Labeling gerecht wird.\\
\\
\textit{Keywords:} automatic label placement, dynamic labeling, graph labeling, interactive labeling
