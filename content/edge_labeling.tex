Im Gegensatz zum Labeling der Knoten werden beim Kantenlabeling einige Einsparung vorgenommen.
Der Grund dafür liegt zum ersten in der Natur von Graphen selbst, da diese für $n$ Knoten bis zu $n^2$ Kanten haben können.
Bei einem großen und dichten Graphen wären das schlichtweg zu viele Labelpositonen, wenn man so aufwendige Verfahren wie bei den Knoten verwendet.

Ein weiterer Grund ist speziell für \texttt{iGraph.js}: Die Daten, die für Kanten zur Verfügung stehen, sind in der Regel nur Integer-Werte.
Da diese Werte ohne Kontext relativ nichtssagend sind, werden Kanten stattdessen mit den Namen der durch sie verbundenen Knoten benannt.
Somit gibt die Kante Auskunft über den Nachbar, selbst wenn dieser selbst nicht zu sehen ist, weil er z. B. außerhalb des Bildschirms ist oder
keine Labelposition für diesen Knoten gefunden werden konnte.

Bei den Kanten wird auf eine Vorsortierung, wie es bei Knoten, verzichtet, vor allem aus Effizienzgründen, aber auch aus fehlender Notwendigkeit, da Kanten tendenziell ergänzend zum Knotenlabeling sind.

Kanten werden mit einer kleineren Schriftgröße gelabelt und erst, wenn das Knotenlabeling abgeschlossen ist und dann noch freie Kapazität vorhanden ist,
das heißt, die festgelegte Maximalanzahl an gezeigten Labels noch nicht überschritten wurden (siehe \hyperref[sec:configuration]{Configuration})
und der \hyperref[subsec:zoom]{Zoom Mode} nicht aktiv ist.

Das Kantenlabeling verzichtet zur Vereinfachung der \hyperref[sec:conflict]{Conflict Detection und Bounds}) auf den gängigen Standard der Kartografie, die Labels
parallel zu der Kante auszurichten.
Aus den oben genannten Gründen ist das Kantenlabeling aber sowieso nebensächlich, weswegen diese Entscheidung für die Performance
und gegen die Ästhetik in diesem Fall gerechtfertigt ist.

\subsubsection{Midpoint}
Der Startpunkt für eine potentielle Labelposition auf Kanten ist grundsätzlich der Mittelpunkt der Kante, das heißt, das Label befindet sich in
gleichmäßigem Abstand zu seinen Knoten und der Mittelpunkt des Labels liegt auf dem der Kante. Die Ausrichtung ist, wie auch bei den Knoten, parallel zur $x$- bzw. $y$-Achse.

\subsubsection{Interpolation}
Weitere potentielle Labelpositionen werden entlang der Kante interpoliert bzw. das Label wird entlang der Kante in gleichmäßigem Abstand verschoben.
Ob dieses Verfahren verwendet werden soll und wenn ja, wieviele Positionen auf der Kante interpoliert werden sollen, sind als \hyperref[subsec:consts]{Magic Constants} in \texttt{consts.js} definiert.
Falls der Mittelpunkt der Kante selbst wieder ein Ergebnis der Interpolation ist, wird er übersprungen, da er zuvor bereits erfolglos als Labelposition getestet wurde und nur dann die Interpolation überhaupt angewandt wird.