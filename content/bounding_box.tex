%comparisions, totally left, right, above, below

Eine Bounding Box ist ein Rechteck um die minimalen und maximalen Ausdehnungen eines Objekts in den $x,y$-Koordinaten,
das somit Grenzen angibt, in dem das Objekt auf jeden Fall vollständig liegt.

Bounding Boxes wurden genutzt um die Bounds, also die Bildschirmgrenze, zu implementieren.
Weiterhin fanden zu Anwendung als Repräsentation der Knoten und der Labels, sodass mit simplen Vergleichen
eine Überdeckung ausgeschlossen werden kann.
Eine weitere Nutzung findet sich im \hyperref[subsec:zoom]{Zoom Mode} zum Ersparen vieler unnötiger Berechnungen.

\subsubsection{Bounds Checking}
Die Bounds werden hier durch eine große Bounding Box repräsentiert.
Für jeden Knoten und für jede Kante muss in jeder Iteration des
Render-Loops kontrollieren werden, ob sie komplett sichtbar sind, das heißt, sich komplett in-bounds befinden.
Außerdem muss dies für generierte Labels bzw. ihre Positionen kontrolliert werden.
Das ist dann der Fall, wenn für das Objekt folgendes gilt:

Sei $a$ das Objekt und $b$ die Bounds. $a,b$ sind Bounding Boxes:
$$    in = a_{xmin} > b_{xmin} \wedge a_{ymin} > b_{ymin} \wedge a_{xmax} < b_{xmax} \wedge a_{ymax} < b_{ymax} $$


Für die Knoten bleibt die Sortierung nach der Größe der Radien auch nach dem Filtern via Bounds Check erhalten,
sodass der wichtigste, derzeit sichtbare Knoten als erstes versucht wird zu labeln.
Um aus dem Versuch, als potenziellen Labelpositionen, tatsächliche Labelpositionen zu machen, muss überprüft werden,
dass diese sich nicht mit Knoten schneiden (oder später mit schon gesetzten, also tatsächlichen, Labelpositionen).

\subsubsection{Label Conflicts}
\label{subsubsec:label_conflict}
Da ein Greedy-Ansatz genutzt wurde, wird, sobald eine potenzielle Labelposition konfliktfrei ($cf$) ist, diese auch sofort genutzt -
auf die Gefahr hin, dass dies Labelpositionen von anderen Objekten blockiert und gegebenenfalls ein besseres Labeling nicht erreicht wird.

Für eine generierte Position, wie in \hyperref[sec:labeling]{Labeling} beschrieben, wird nun zur Konfliktermittlung eine Bounding Box erzeugt, also das Label "eingerahmt".
Für diese Bounding Box wird jetzt kontrolliert, ob sie sich mit keiner Bounding Box von allen sichtbaren Knoten und keiner Bounding Box von gegebenenfalls schon erzeugten Labels überdeckt.
Das heißt, für alle Vergleich muss gelten: Die Bounding Box des potenziellen Labels ist total links von, rechts von, über oder unter der liegen mit der sie verglichen wird (also des Knotens oder tatsächlichen Labels):

Sei $a$ die potenzielle Bounding Box und $B$ die Menge der Bounding Boxes von Knoten und tatsächlichen Labels:
$$ cf \leftrightarrow \forall b \in B: a_{xmax} < b_{xmin} \vee a_{xmin} > b_{xmax} \vee a_{ymax} < b_{ymin} \vee a_{ymin} > b_{ymax} $$

So können sukzessive potenzielle Labelpositionen überprüft und, falls konfliktfrei, zu tatsächlichen Labels umgewandelt werden,
die dann im weiteren Verlauf selbst Berücksichtigung bei der Konfliktermittlung finden.