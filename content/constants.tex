%\texttt{consts.js} for all magic constants, easily configurable

Ursprünglich zum vereinfachten internen Testen und zur Strukturierung des Codes gedacht, ist
eine Sammlung der Magic Constants gegebenenfalls auch für externe Nutzer interessant. Viele Parameter, die über das GUI
nicht verstellt werden können, sind hier konfigurierbar, beispielweise:
\begin{itemize}
    \item Schriftart und -farbe
    \item Interpolationsschritt für Kanten (siehe \hyperref[subsec:labeling_edges]{Kantenlabeling})
    \item Ineffizienzgrenzen der Teilregionen des Quadtree (siehe \hyperref[subsec:quadtree]{Quadtree}).
    \item Parameter der Spirale (siehe \hyperref[subsubsec:spiral]{Spiral-Model})
    \item Grenzwerte für die Performance Modes (siehe \autoref{fig:perf_modes})
\end{itemize}

Änderung an den Werten können immer dazu führen, dass das Programm gar nicht funktioniert oder zumindest nicht wie erwartet.
Für die Nutzung von gebräuchlichen Werten sollten allerdings maximal die Performance schlechter werden, Funktionalität aber erhalten bleiben.

Möglicherweise gibt es auch noch bessere Einstellungen hinsichtlich der Performance, die visuell gleichermaßen ansprechend sind wie die Standardeinstellungen.