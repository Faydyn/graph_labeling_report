Graphen sind wohl die mit am meisten genutzte Datenstruktur der Informatik.
Durch sie lassen sich viele verschiedene Sachverhalte modellieren.
Mit \texttt{iGraph.js} existiert als praktische Grundlage ein Tool zur interaktiven und dynamischen Visualisierung von Graphen von Tominski et al.\cite{interactive}

Das Problem ist, dass ohne die Namen der Knoten und Kanten dem Graph recht wenig Bedeutung abgewonnen werden kann,
es fehlt Kontext. Der Kontext kann durch ein Labeling des Graphen klarer werden. Dazu sind allerdings zwei Dinge zentral:

Erstens sollte das Labeling visuell ansprechend sein, das heißt, es muss intuitiv klar sein, welches Label zu welchem Knoten oder zu welcher Kante gehört
und es sollten keine Überdeckungen entstehen.

Zweitens sollte das Labeling performant sein, da selbst die schönste Anordnung von Labels nicht nützlich ist in einem dynamischen und interaktiven Umfeld, wenn sie zu lange zum Errechnen braucht.
Den bestmöglichen Kompromiss aus Labeling Qualität und Performance zu finden, ist die Kernaufgabe dieses Projektes.

Dazu haben wir eine Labeling Pipeline implementiert:
\begin{enumerate}
    \item Koordinaten der Knoten/Kanten updaten (von \texttt{iGraph.js})
    \item Potentielle Labelpositionen generieren
    \item Konfliktermittlung
    \item Tatsächliche Labelposition festlegen und speichern
    \item Visualisierung
\end{enumerate}

Natürlich ist diese nicht ganz so statisch und sukzessiv wie hier vereinfacht beschrieben.
Zum Beispiel finden Schritt 2 und 3 verschränkt statt, also werden nicht erst alle potentiellen Positionen generiert und dann nach Konflikten gesucht.
Stattdessen finden sie abwechseln statt.

Dies spiegelt den gierigen Ansatz wieder, der dieser Implementation zu Grunde liegt. Dadurch soll die Kernaufgabe des Projektes umgesetzt werden,
also schnell möglichst gute Labelpositionen zu finden. Dies schien die sinnvollste Option in Anbetracht der Tatsache, dass das Labeling Problem NP-Hard ist.\cite{nphard}

Es sei abschließend gesagt, dass diese Pipeline und die verwendeten Techniken und Methoden nicht ausschließlich für Graph Labeling genutzt werden müssen.
Beispielsweise wäre auch eine Nutzung für Radare von etwa Schiffen oder Flugzeugen vorstellbar, bei denen andere Schiffe/Flugzeuge mit Namen versehen werden sollen.
Formal genommen ist das ja ein Graph ohne Kanten - sollte nur darstellen, dass die Techniken und Algorithmen sehr generisch sind.

Als hauptsächliche theoretische Grundlage diente zur Spezifikation der Pipeline sowie der Labeling Verfahren und Konfliktermittlung eine Arbeit von Luboschik et al.\cite{main}, in der das Vorgehen beschrieben wird.